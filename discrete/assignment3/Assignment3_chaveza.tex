\documentclass{article}
\usepackage{amsfonts}
\usepackage{amssymb}
\usepackage{graphicx}
\begin{document}
\hfill Alejandro Chavez

\hfill Assignment 3 - Discrete Mathematics

\hfill \today\\

\begin{center}\begin{large}Assignment 3\end{large}\end{center}\begin{itemize}
	\item
		Section 1.7
	\begin{itemize}
		\item
			2)
			Prove: $\lnot \exists x\{0<x<1000,\; x=\sqrt[3]{c},\; x=a^{3}+b^{3}$ where $a$, $b$, and $c \in \mathbb{N} \}$\\
      \begin{center} \begin{tabular}{c|l}
      List of perfect cubes & Possible values for $a^3$ and $b^3$ \\ \hline
      1 & none\\
      8 & $1+1=2<8$\\
      27 & $8+8=16<27 $\\
      64 & $27+27=54<64$\\
      125 & $64+64=128>125,\; 64+27=91<125$\\
      216 & $125+125=250>216,\; 125+64=189<216$\\
      343 & $216+216=432>343,\; 216+125=341<343$\\
      512 & $343+343=686>512,\; 343+216=559>512,\; 216+216=432<512$\\
      729 & $512+512=1024>729,\; 512+343=855>729,\; 343+343=686<729$\\
      \end{tabular} \end{center}
      $\therefore \lnot \exists x\{0<x<1000,\; x=\sqrt[3]{c},\; x=a^{3}+b^{3}$ where $a$, $b$, and $c \in \mathbb{N} \}$\\
      Through a very exaustive means.
    \item
      6)
      Prove: $x$, $a$, and $b\in \mathbb{N};\; \exists x\{ x=a+b,\; a<x,\; b<x \}$\\
      1+2=3\\
      $\therefore x$, $a$, and $b\in \mathbb{N};\; \exists x\{ x=a+b,\; a<x,\; b<x \}$\\
      This proof is constructive.
    \item
      10)
      Prove: Product of two of the the numbers a, b, and c is non-negative.\\
      Suppose a is positive.\\
        Suppose b is positive. $a*b=$ a non-negative.\\
        If b is negative,\\
          suppose c is positive. $a*c=$ a non-negative.\\
          If c is negative, $b*c=$ a non-negative.\\
      If a is negative,\\
        suppose b is negative. $a*b=$ a non-negative.\\
        If b is positive,\\
          suppoce c is negative. $a*c=$ a non-negative.\\
          If c is positive, $b*c=$ a non-negative.\\
      If a, b, or c $=0$, zero times anything is a non-negative.\\

      This method of proof is non-constructive.
    \item
      20)
      prove: $x^{2}+\frac{1}{x^{2}}\ge 2$\\
      $x^{2} + \frac{1}{x^{2}}-2\ge 0$\\
      $(x-\frac{1}{x})^{2}\ge 0$\\
      If $x-1/x=0$\\
      \hspace*{2em} $(x-\frac{1}{x})^{2}=0$\\
      If $x-1/x<0$\\
      \hspace*{2em} $(x-\frac{1}{x})^{2}>0$\hfill \hspace*{2em} since any negative real number squared is positive.\\
      If $x-1/x>0$\\
      \hspace*{2em} $(x-\frac{1}{x})^{2}>0$\hfill \hspace*{2em} since any positive real number squared can never equal to zero or be non-negative.\\
    \item
      26) $a^{2}=b$\\
      \begin{tabular}{c|c}
      if the least signficant digit of a & least significant digit of b \\ \hline
      0 & 0 \\
      1 or 9 & 1 \\
      2 or 8 & 4 \\
      3 or 7 & 9 \\
      4 or 6 & 6 \\
      5 & 5 \\
      \end{tabular}
  \end{itemize}
  \item
    Review Exercises
  \begin{itemize}
    \item
      4)
      \begin{itemize}
        \item a)
          Two propositions are logically equivalent when they have the same truth values when their propositional variables have the same truth values.
        \item b)
          You can show compund propositions are equivalent by truth tables, deductive resoning, and contradiction.
        \item c) \\
        (1) $\lnot p \lor (r \to \lnot q)$ \hfill Given\\
        (2) $\lnot p \lor \lnot r \lor \lnot q$ \hfill Material Implication(1)\\
        (3) $\lnot p \lor \lnot q \lor \lnot r$ \hfill Associativity(2)\\

        \begin{tabular}{c|c|c||c|c}
          p & q & r & $\lnot p \lor (r \to \lnot q)$ & $\lnot p \lor \lnot q \lor \lnot r$ \\ \hline
          T&T&T&F&F\\ 
          T&T&F&T&T\\ 
          T&F&T&T&T\\ 
          T&F&F&T&T\\ 
          F&T&T&T&T\\ 
          F&T&F&T&T\\ 
          F&F&T&T&T\\ 
          F&T&T&T&T\\ 
        \end{tabular}
      \end{itemize}
    \item 6)\\
    $\forall P(x)$\\
    $\exists P(x)$\\
    $\lnot \forall P(x)$\\
    $\lnot \exists P(x)$\\
    \item 14) 
    A constructive proof will show you explicitly the existance of a case where what ever you are trying to proof is true, while a noncontructive proof will show that a case where what you are trying to prove exists through hypothetical situations.\\
    Constructive proof:\\
      prove there exists $a$ where $b+a=b$\\
      $3+0=3$\\
    Nonconstructive proof:\\
      prove there exists integer $a$ where $a/2=b$ and $b$ is integer\\
      suppose $a$ is even.\\
      $a = 2c$ where c is an integer.\\
      $2c/2 = c = b$\\
  \end{itemize}
  \item 
  Supplementary Exercises
  \begin{itemize}
    \item
    4)
    \begin{itemize}
      \item
      a)\\
      If I will drive to work, then it is raining today.\\
      If I will not drive to work, then it is not raining today.\\
      If it is not raining today, then I will not drive to work.
      \item
      b)\\
      If $x\ge 0$, then $|x|=x$\\
      If $x< 0$, then $|x|\ne x$\\
      If $|x|\ne x$, then $x<0$
      \item
      c)\\
      If $n^{2}>0$, then $n>3$\\
      If $n^{2}\le 0$, then $n\le 3$\\
      If $n\le 3$, then $n^{2}\le 0$
    \end{itemize}
  \item
  6)\\
  Inverse of Inverse: \hfill $p\to q$\\
  Inverse of Converse: \hfill $\lnot q \to \lnot p$\\
  Inverse of Contrapositive: \hfill $q \to p$
  \item
  14)
  \begin{itemize}
    \item
    a) $\exists xP(x)$
    \item
    b) $\lnot \forall x P(x)$
    \item
    c) $\forall yQ(y)$
    \item
    d) $\forall x\forall y P(x)Q(y)$
    \item
    e) $\exists y\lnot Q(y)$
  \end{itemize}
  \item
  18) $\forall y \lnot \exists xG(x,y)\{x>3\}$
  \item
  30) If it is given that P(x) and P(y) is true for values x and y, then we can simplify what is given by saying that P(x) is true for x and we can also say that P(y) is true for y.
  \item
  34) given: $n$, $n\in \mathbb{Z}^{+}$\\
  prove: $m^{2}\le n<(m+1)^{2}$\\
  if $n=a^{2}$ where $a\in \mathbb{Z}^{+}$\\
  $\;\; m^{2}\le a^{2}<(m+1)^{2}$\\
  $m=a$\\
  If n is not a perfect square, then $m^{2}$ would be the closest smaller perfect square, and since there are no perfect squares right next to each other one the integer line, there will always space for an $n$ in between perfect squares($m^{2}$ and $(m+1)^2$). 
  \end{itemize}
  \item
  Section 2.1
  \begin{itemize}
    \item
    2)
    \begin{itemize}
    \item
    a) $\{x\in \mathbb{N}|\;x=3n,\;n\in \mathbb{N},\;n\ge 4\}$
    \item
    b) $\{x\in \mathbb{Z}|\;-3\le x\le3\}$
    \item
    c) $\{l\in Alphabet|\;l$ is between 'l' and 'q' in the sequence $Alphabet\}$
    \end{itemize}
    \item
    4)
    \begin{itemize}
    \item
    $B\subset A$\\
    \item
    $C\subset A$\\
    \item
    $C\subset D$
    \end{itemize}
    \item
    8)
    \begin{itemize}
    \item
    a) T
    \item
    b) T
    \item
    c) F
    \item
    d) T
    \item
    e) T
    \item
    f) T
    \item
    g) T
    \end{itemize}
    \item
    16) $B=\{A,x\}$, $A=\{x\}$
    \item
    18) \begin{itemize}
    \item
    a) $0$
    \item
    b) $1$
    \item
    c) $2$
    \item
    d) $3$
    \end{itemize}
    \item
    20) Yes, because if they have the same power set, they have the same elements.
    \item
    22) \begin{itemize}
    \item
    a) no
    \item
    b) no
    \item
    c) No!
    \item
    d) yes
    \end{itemize}
    \item
    30)\\
    Suppose $A=\{1\}$ and $B=\{2\}$\\
    $A$x$B=\{(1,2)\}$\\
    $B$x$A=\{(2,1)\}$\\
    $A\ne B$
  \end{itemize}
  \item
  Section 2.2
  \begin{itemize}
  \item
  2) \begin{itemize}
  \item
  a)
  $A\cap B$
  \item
  b)
  $A-B$
  \item
  c)
  $A\cup B$
  \item
  d)
  $A^{c}\cup B^{c}$
  \end{itemize}
  \item
  4)\begin{itemize}
    \item
    a) $\{a,b,c,d,e,f,g,h\}$
    \item
    b) $\{a,b,c,d,e\}$
    \item
    c) $\emptyset$
    \item
    d) $\{f,g,h\}$
  \end{itemize}
  \item
  12)\\
  (1)$A\cup (A\cap B)$\\
  (2)$(A\cap B)\in A$\hfill Definition of Intersection (1)\\
  (3)$A\cup x=A$, where $x\in A$\hfill Definition of Union\\
  (4)$A\cup (A\cap B) = A$\hfill (1),(2),(3)
  \item
  16)\begin{itemize}
    \item
      a)\\
      $\forall x\in (A\cap B)$\\
      $x\in A$ and $x\in B$\\
      and since $x\in A$\\
      $(A\cap B)\subseteq A$
    \item
      b)\\
      $\forall x\in A$\\
      $x\in (A\cup B)$\\
      $\therefore A\subseteq (A\cup B)$
    \item
      c)\\
      $\forall x\in A$\\
      if $x\in B$\\
      then $x\notin (A-B)$\\
      if $x\notin B$\\
      then $x\in (A-B)$\\
      $\therefore A-B\subseteq A$
    \item
      d)\\
      (1)$A\cap (B-A) = A\cap (B\cap A^{c})$\\
      (2)$A\cap A^{c}\cap B$\hfill Association (1)\\
      (3)$\emptyset \cap B$\hfill Complement(2)\\
      (4)$\emptyset$\hfill Domination(3)\\
    \item
      e)\\
      (1)$A\cup (B-A)$\\
      (2)$A\cup (B\cap A^{c})$\\
      (3)$(A\cup B)\cap (A\cup A^{c})$\hfill Distribution(2)\\ 
      (4)$(A\cup B)\cap \emptyset$\hfill Complemet(3)\\
      (5)$A\cup B$\hfill Identity(4)\\
  \end{itemize}
  \item
  18)\begin{itemize}
    \item
      a)\\
      $\forall x\in (A\cup B)$\\
      $x\in (A\cup B\cup C)$\\
      $\forall y\in (A\cup B\cup C)$\\
      $y\in (A\cup B)$ or $y\notin (A\cup B)$\\
      $\therefore (A\cup B)\subseteq (A\cup B\cup C)$
    \item
      b)
    \item
      c)
    \item
      d)
    \item
      e)
  \end{itemize}
  \item
  30)
  \item
  36)
  \end{itemize}
\end{itemize}
\end{document}
