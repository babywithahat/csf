\documentclass[10pt,a4paper]{article}
\usepackage{amssymb}
\begin{document}
\hfill Alejandro Chavez

\hfill Assignment 2 - Discrete Mathematics

\hfill \today\\

\begin{itemize}
  \item Section 1.4
    \begin{itemize}
      \item Problem 2
        \begin{itemize}
          \item a)
            There exists a real number x such that for all real numbers y, $xy = y$ is true.
          \item b)
            For all real numbers x and for all real numbers y, if x is greater than or equal to zero and y is greater than or equal to zero, then $xy\ge0$ is true.
          \item c)
            For all real numbers x and for all real numbers y there exists a real number z where $x = y + z$ is true.
        \end{itemize}
      \item Problem 4 
        \begin{itemize}
          \item a)
            There exists a student in my class that has taken a computer science course in my school.
          \item b)
            There exists a student in my class that has taken all the computer science courses in my school.
          \item c)
            All the students in my class have taken a computer science course in my school.
          \item d)
            There exists a computer science course in my school that all the students in my class have taken.
          \item e)
            All the computer science courses in my school have been take by a student in my class.
          \item f)
            All the students in my class have taken all the computer science courses in my school
         \end{itemize}
      \item Problem 8
        \begin{itemize}
          \item a)
           $\exists x\exists yQ(x,y)$ 
          \item b)
            $\lnot \exists x\exists yQ(x,y)$
          \item c)
            $\exists xQ(x,(Jeopardy,Wheel\:of\:Fortune))$
          \item d)
            $\forall y\exists xQ(x,y)$
          \item e)
            $\exists xQ(x\ge 2, Jeopardy)$
        \end{itemize}
    \end{itemize}
  \item Section 1.5
    \begin{itemize}
      \item Problem 2
        \begin{itemize}
          \item
          The argument is valid by Modus Tollens. We can conclude the conclusion to be true if the premise is true.
        \end{itemize}
      \item Problem 4
        \begin{itemize} 
          \item a)
            Simplificiton
          \item b)
            Disjunctive Syllogism
          \item c)
            Modus Ponens
          \item d)
            Addition
          \item e)
            Hypothetical Syllogism
        \end{itemize} 
      \item Problem 10
      \begin{itemize}
        \item a)
          "I did not play hockey yesterday." \hfill Modus Tollens
        \item b)
          "If worked last Friday then Friday was sunny"
        \item c)
          "Dragonflies have six legs"\hfill Modus Ponens\\
          "Spiders are not insects"\hfill Modus Tollens
        \item d)
          "Homer is not a student" \hfill Modus Tollens
        \item e)
          "Tofu does not taste good"\hfill Modus Ponens\\
          "I do not eat healthy food"\hfill Modus Tollens
        \item f)
      \end{itemize}
    \end{itemize}
  \item Section 1.6
    \begin{itemize}
      \item Problem 2\\
        $a = 2x,\:b = 2y$\hfill Where a, b, x and y are integers.\\
        $a + b = 2x + 2y = 2(x + y)$
      \item Problem 4\\
        $a + b = 0$ \hfill Where a and b are even integers and thus ...\\
        $a = 2x$    \hfill ...x is an integer.\\
        $2x = -b$\\
        $x = -b/2$ \hfill Since x is an integer, b has to be an even integer.
      \item Problem 10\\
        $a = w/x$ \hfill Where $a\in \mathbb{Q}$ , and w and z $\in \mathbb{Z}$.\\
        $b = y/z$ \hfill Where $b\in \mathbb{Q}$ , and y and z $\in \mathbb{Z}$.\\
        $ab = (wy)/(xz)$ \hfill By definition, ${ab} \in \mathbb{Q}$.
      \item Problem 18
        \begin{itemize}
          \item a) Contraposition:\\
            Prove:If $n$ is not even, then $n$ is not an integer or $3n+2$ is odd.\\
            $n = 2a + 1$ \hfill if n is an odd integer, then and a $\in \mathbb{Z}$\\
            $3(2a + 1) +2 = 6a + 5$ \hfill Any integer multiplied by an even number and added by an odd number is odd, therefore $3n+2$ is odd.\\
          \item b) Contradiction:\\
            If $n$ is not even, then $n$ is an integer and $3n+2$ is even.\\
            Given: $n$ is not even\\
            Prove: $n$ is not an integer or $3n+2$ is odd.\\
            $n = 2a + 1$ \hfill if n is an odd integer, then and a $\in \mathbb{Z}$\\
            $3(2a + 1) +2 = 6a + 5$ \hfill Any integer multiplied by an even number and added by an odd number is odd, therefore $3n+2$ is odd.\\
        \end{itemize}
      \item Problem 20\\
        1 is a positive integer.\\
        $1^{2}=1$\\
        $1\ge 1$\\
        Direct Proof.
    \end{itemize}
\end{itemize}

\end{document}
